\section*{Research program and fit to AE0075}

This proposal is written for Imperial College London’s AE0075 studentship on novel algorithms for load rebalancing of adaptive simulations on heterogeneous clusters, with Nektar++ as the implementation and evaluation platform.
The goal is simple and measurable: reduce timestep wall time while avoiding wasted rebalances and migration churn.

Adaptive high-order simulation does not have a fixed “workload shape”. As the simulation adapts, the expensive work moves around the domain. On heterogeneous CPU/GPU clusters, “equal work” is not equal element counts, and synchronized phases can make end-to-end runtime depend on the slowest partition. That is the real problem AE0075 is pointing at.

We start from what we already know from measured signals: cost can rise sharply with local approximation order, clustered refinement can create time hotspots, solver-heavy phases can dominate at scale, and periodic schedules can waste rebalance calls when the mapping has not meaningfully changed. Taken together, this says the same thing: rebalancing should not be a fixed schedule. It should be a decision made from measurable quantities.

What we do next is organized around that decision. We will (1) make the workload and imbalance visible in wall-time terms (not just counts), (2) measure the real overhead of repartition+migration+coordination, and (3) trigger rebalancing only when the predicted savings exceed the predicted cost, with an explicit cap on how much we move so the method stays stable and doesn’t thrash.

Success is not a single good plot. A result counts only if it reduces \(t_{\text{step}}\) against clear baselines (static mappings and periodic “rebalance every \(k\)”), while keeping overhead bounded, reducing wasted/no-op events, and preserving solution acceptability (QoI tolerance), across a sweep of regimes (hotspot severity, adapt cadence, and heterogeneity).
