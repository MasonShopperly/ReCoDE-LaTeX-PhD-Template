\section*{Research aims, hypotheses, and evaluation logic}

This project treats load rebalancing as a measurable decision within adaptive high-order simulation on heterogeneous clusters: given that the cost distribution changes over time, decide \emph{when} to rebalance and \emph{how much} to migrate so that timestep wall time decreases without creating migration churn.

Rather than a long list of features, the work is organized around three concrete questions:
\begin{enumerate}
  \item \textbf{Savings:} if we change the mapping, how much wall time is saved (phase-aware)?
  \item \textbf{Cost:} what overhead do we pay to repartition, migrate, and coordinate?
  \item \textbf{Decision:} under what conditions is a rebalance net-positive, and how do we bound movement?
\end{enumerate}

These questions become falsifiable hypotheses:
\begin{itemize}
  \item \textbf{H1 (savings is predictable):} a cost-weighted mapping calibrated to order sensitivity reduces slowest-part throttling compared to count-based baselines, especially under clustered refinement.
  \item \textbf{H2 (cost matters):} in realistic adaptive cadence, periodic policies waste time in regimes where mappings change weakly (including no-op / near-no-op events) because overhead exceeds benefit.
  \item \textbf{H3 (a gate beats a schedule):} a triggered rule that rebalances only when \(\Delta T_{\text{benefit}}>\Delta T_{\text{cost}}\), combined with a migration budget, matches or improves wall time versus periodic rebalancing while reducing event count and preventing churn.
\end{itemize}

\noindent\textbf{Evaluation logic.}
We compare against simple baselines (static count-based, static cost-weighted, periodic cadence) across controlled regimes (hotspot severity, adaptive cadence, heterogeneity).
A result counts only if it reduces \(t_{\text{step}}\) while keeping overhead bounded, limiting migration volume, and maintaining solution acceptability (QoI tolerance).
