\section*{Research aims, hypotheses, and evaluation logic}

The introduction frames a concrete decision problem: in an adaptive simulation whose cost distribution evolves over time, \emph{when} should repartitioning/migration occur and \emph{how much} state should move, so that wall time decreases without triggering churn?
We make this decision measurable by pairing each research aim with a falsifiable hypothesis and a controlled evaluation protocol.

\vspace{0.4em}
\noindent\textbf{Aim A1 (model benefit):} predict how wall time changes when expensive regions are redistributed across ranks/devices, in a phase-aware way.
\newline\textbf{Hypothesis H1:} cost-weighted balance (informed by calibrated order sensitivity) reduces slowest-part throttling relative to count-based balance, especially under clustered refinement.

\vspace{0.35em}
\noindent\textbf{Aim A2 (model cost):} quantify and predict rebalancing overhead (repartition + migration + coordination) as a function of migration volume and regime.
\newline\textbf{Hypothesis H2:} for realistic adaptive cadence, periodic rebalancing wastes time in regimes where the mapping changes weakly, including no-op / near-no-op events.

\vspace{0.35em}
\noindent\textbf{Aim A3 (make the decision):} design a triggered policy that executes a rebalance only when predicted net gain is positive, with bounded migration to prevent churn.
\newline\textbf{Hypothesis H3:} a trigger rule using \(\Delta T_{\text{benefit}} > \Delta T_{\text{cost}}\) plus a migration budget matches or improves wall time compared to periodic policies while reducing event count and stabilizing behavior across regimes.

\vspace{0.55em}
\noindent\textbf{Evaluation matrix (what “success” means).}
\begin{tabularx}{\linewidth}{@{}p{0.18\linewidth}X p{0.18\linewidth} p{0.18\linewidth}@{}}
\toprule
\textbf{Question} & \textbf{Test / protocol} & \textbf{Baseline} & \textbf{Pass condition} \\
\midrule
Does imbalance throttle timestep time? &
Hold physics case fixed; induce clustered refinement / synthetic hotspots; measure phase-aware wall time and slowest-part throttling. &
Static / count-based partition. &
Reduced \(t_{\text{step}}\) with lower \(\rho\) in imbalance-sensitive phases. \\
\addlinespace
Does periodic rebalancing waste time? &
Run identical adaptive cadence under periodic schedule; track overhead and no-op (0-move) events. &
Periodic every \(k\). &
Non-trivial fraction of time spent in overhead and/or no-op events in weak-change regimes. \\
\addlinespace
Does triggered + bounded migration win robustly? &
Sweep regimes (heterogeneity, hotspot severity, adapt cadence); compare wall time + overhead + event count. &
Periodic every \(k\), static mapping. &
Equal-or-better \(t_{\text{step}}\) with fewer events, bounded overhead, stable behavior. \\
\bottomrule
\end{tabularx}

\vspace{0.35em}
\noindent The next sections define the core quantities used throughout (\S\,Definitions/Notation), position the work in the literature, and then develop the technical approach and evaluation plan in full.
