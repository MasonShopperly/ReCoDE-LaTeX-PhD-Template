\section*{Research program and fit to AE0075}

This proposal responds directly to Imperial College London’s AE0075 studentship on \emph{novel algorithms for load rebalancing of adaptive simulations on heterogeneous clusters}, using Nektar++ as the implementation and evaluation platform.
The core objective is practical and measurable: reduce timestep wall time in adaptive high-order simulation without triggering unnecessary repartitioning, migration churn, or instability across heterogeneous regimes.

\vspace{0.35em}
\noindent\textbf{What AE0075 is asking for (distilled).}
Adaptive simulations change where computational effort is concentrated; heterogeneity changes what “equal work” means; and synchronized phases can make end-to-end time hostage to the slowest partition.
AE0075 therefore demands algorithms that decide \emph{when} to rebalance and \emph{how much} to move, under realistic overheads, not just a better static partition.

\vspace{0.35em}
\noindent\textbf{What is already established (signals that motivate the approach).}
Our preliminary measurements show that (i) local approximation order can drive strongly non-uniform cost, (ii) clustered refinement produces time hotspots that inflate max/mean imbalance, (iii) globally coupled phases can dominate at scale, and (iv) periodic schedules can waste rebalance events when mappings change weakly.
These are exactly the failure modes AE0075 cares about: naive proxies and fixed schedules break down under adaptive, heterogeneous, synchronized workloads.

\vspace{0.35em}
\noindent\textbf{What we will build (the algorithmic objects).}
\begin{itemize}
  \item \textbf{A phase-aware load proxy:} a calibrated mapping from local discretization state (e.g., order distribution and measured phase timers) to predicted contribution to \(t_{\text{step}}\).
  \item \textbf{An overhead proxy:} a measurable model for the cost of repartitioning, migration, and coordination as a function of movement volume and system regime.
  \item \textbf{A decision policy:} a triggered rule that executes rebalancing only when predicted net gain is positive, coupled to a migration budget (bounded movement) to prevent churn.
  \item \textbf{A Nektar++ integration story:} a concrete hook-point map that makes the adapt \(\leftrightarrow\) repartition \(\leftrightarrow\) migrate lifecycle implementable and testable.
\end{itemize}

\vspace{0.35em}
\noindent\textbf{How we will validate (what counts as success).}
\begin{itemize}
  \item \textbf{Primary metric:} reduced timestep wall time \(t_{\text{step}}\) under matched physics/QoI tolerance.
  \item \textbf{Required accounting:} explicit overhead (repartition+migration+coordination time), number of events (including no-ops), and migration intensity (moved volume/time).
  \item \textbf{Baselines:} static count-based mapping, static cost-weighted mapping, and periodic “rebalance every \(k\)” schedules.
  \item \textbf{Regime coverage:} controlled sweeps over hotspot severity, adaptive cadence, and heterogeneity factors to ensure robustness rather than one-off wins.
\end{itemize}

\noindent A result is only credited if it reduces \(t_{\text{step}}\) while keeping overhead bounded and preventing churn, across multiple regimes—not just one favorable case.
