\section*{Related work and positioning (what exists, and what is missing)}

\noindent\textbf{High-order CFD foundations and platform.}
This work sits on the spectral element / spectral-\(hp\) CFD foundation (e.g., \cite{Patera1984SpectralElement,KarniadakisSherwin2005}) and uses Nektar++ as the target implementation platform, where algorithmic contributions are naturally testable in a mature solver stack (e.g., \cite{Cantwell2015NektarPP,Cantwell2020NektarPP}; see also \cite{NektarWebsite}).

\vspace{0.35em}
\noindent\textbf{Partitioning and dynamic load balancing.}
Graph partitioning is a standard baseline for distributing irregular workloads (e.g., multilevel partitioning in \cite{KarypisKumar1998METIS}).
For problems whose cost distribution changes over time, dynamic load balancing frameworks explicitly support remapping and repartitioning decisions (e.g., \cite{Devine2002Zoltan}).
For adaptive mesh refinement at scale, modern AMR infrastructures demonstrate that rebalancing must be treated as a first-class algorithmic component to remain scalable (e.g., \cite{Burstedde2011p4est}).

\vspace{0.35em}
\noindent\textbf{Heterogeneity-aware scheduling and overhead.}
On heterogeneous clusters, “equal work” depends on device throughput and on which phases dominate time-to-solution.
Heterogeneous scheduling models (e.g., \cite{Topcuoglu2002HEFT}) and heterogeneous runtime systems (e.g., \cite{Augonnet2011StarPU}) highlight a key point: performance depends not only on mapping, but also on \emph{decision overhead} and the stability of that decision over time.
Classic dynamic load balancing work similarly emphasizes that policies must be judged by end-to-end wall time, not by structural balance alone (e.g., \cite{Cybenko1989DynamicLoadBalancing}).

\vspace{0.35em}
\noindent\textbf{Where the gap is (what this proposal contributes).}
Load balancing exists, but adaptive spectral/\(hp\) introduces a particularly sharp combination that is often under-modeled in practice:
(i) per-element cost grows strongly and nonlinearly with local approximation order \(P\),
(ii) rebalancing overhead (repartition + migration + coordination) can dominate when triggered too often or when the mapping changes weakly,
and (iii) heterogeneity means a good mapping is phase- and device-dependent.
This proposal addresses that gap by explicitly modeling \(\Delta T_{\text{benefit}}\) and \(\Delta T_{\text{cost}}\), using them to gate \emph{when} rebalancing occurs, and bounding \emph{how much} migration occurs to prevent churn.

\vspace{0.35em}
\noindent\textbf{Short-term literature expansion (planned, mapped to aims).}
Next additions will focus on: (a) hypergraph repartitioning and migration-minimizing variants (to support bounded movement), and (b) performance modeling papers that separate imbalance-sensitive phases from globally coupled phases. Each new citation will be paired with a one-sentence “why it matters here” and tied to a measurable quantity used in the evaluation plan.
