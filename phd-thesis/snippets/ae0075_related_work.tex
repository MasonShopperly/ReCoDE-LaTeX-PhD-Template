\section*{Related work and positioning (what exists, and what is missing)}

\noindent\textbf{High-order foundations (why \(P\) matters).}
Spectral element and spectral/\(hp\) methods achieve high accuracy with strong element locality, but their runtime is sensitive to polynomial order and operator structure.
This work builds on classical spectral element foundations and the spectral/\(hp\) CFD literature (e.g., \cite{Patera1984SpectralElement,KarniadakisSherwin2005}).

\vspace{0.35em}
\noindent\textbf{Target platform: Nektar++.}
Nektar++ provides a mature open-source spectral/\(hp\) framework with scalable operators and solvers, making it a credible vehicle for publishable algorithm work in “real code”
(e.g., \cite{Cantwell2015NektarPP,Cantwell2020NektarPP}; see also \cite{NektarWebsite}).

\vspace{0.35em}
\noindent\textbf{Partitioning and dynamic load balancing.}
Graph-based partitioning and dynamic load balancing are well-established in HPC practice, with widely used baselines such as METIS-style multilevel partitioning \cite{KarypisKumar1998METIS} and software frameworks targeting dynamic applications (e.g., Zoltan \cite{Devine2002Zoltan}).
In the AMR ecosystem, scalable parallel adaptivity and repartitioning are also treated as first-class problems (e.g., p4est \cite{Burstedde2011p4est}).

\vspace{0.35em}
\noindent\textbf{Where the gap is (what we contribute).}
The gap is not “repartitioning exists,” but the combination of constraints that appear in adaptive high-order CFD on heterogeneous clusters:
(i) per-element cost grows strongly with local approximation order (so count-based balance is unreliable),
(ii) rebalancing has nontrivial overhead (repartition + migration + coordination) that can dominate when triggered too often or when mappings change weakly, and
(iii) heterogeneity means “equal work” depends on device throughput and which timestep phases dominate.
This proposal explicitly models both \emph{benefit} and \emph{cost}, then uses those models to decide \emph{when} to rebalance and \emph{how much} to migrate (bounded movement).

\vspace{0.35em}
\noindent\textbf{Short-term literature expansion plan (next research-mode push).}
We will next add citations specific to: (a) heterogeneous scheduling models (CPU/GPU), (b) repartition+migration overhead modeling, and (c) high-order adaptivity case studies.
Each new citation will be accompanied by a single sentence stating \emph{what it contributes here} and mapped to a measurable quantity in the Definitions/Notation sections.
We will also anchor the narrative directly to the studentship framing \cite{ImperialAE0075Studentship} and the supervisors’ group context (e.g., \cite{SherwinLabWebsite}).
