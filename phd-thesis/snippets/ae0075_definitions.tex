\section*{Operational definitions (metrics used throughout)}

To keep claims falsifiable, the proposal uses a small set of operational metrics that recur in the aims, baselines, and evaluation.
A “win” is not a plot that looks nicer: it is reduced wall time with bounded overhead and stable behavior across regimes.

\vspace{0.35em}
\noindent\textbf{Primary performance metric.}
\begin{itemize}
  \item \textbf{Timestep wall time} \(t_{\text{step}}\): average wall time per timestep (primary objective).
\end{itemize}

\vspace{0.35em}
\noindent\textbf{Imbalance and effective utilization.}
\begin{itemize}
  \item \textbf{Imbalance ratio} \(\rho=\frac{\max(\text{rank time/load})}{\text{mean}(\text{rank time/load})}\), where \(\rho=1\) is perfectly balanced.
  \item \textbf{Effective efficiency} \(\eta \approx \frac{\text{mean}}{\max}\), where \(\eta \approx 1\) is ideal.
\end{itemize}

\vspace{0.35em}
\noindent\textbf{Rebalancing overhead and churn.}
\begin{itemize}
  \item \textbf{Event count:} number of rebalances, including no-op events (0 moves).
  \item \textbf{Migration intensity:} elements moved and migration time (as a direct proxy for churn and overhead).
  \item \textbf{Overhead time:} repartition + migration + coordination time attributable to rebalancing.
\end{itemize}

\vspace{0.35em}
\noindent\textbf{Solution acceptability.}
\begin{itemize}
  \item \textbf{QoI tolerance:} a fixed criterion defining acceptable solution change under rebalancing; policies are only compared within this tolerance.
\end{itemize}

