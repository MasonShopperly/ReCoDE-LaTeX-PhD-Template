\section*{Operational definitions (metrics used throughout)}

The proposal is measurement-driven: the claims we make are expressed in terms of a small set of operational metrics that appear repeatedly in the aims and evaluation matrix.
The intent is to prevent “hand-wavy speedups”: every win must show up as reduced wall time with bounded overhead.

\vspace{0.35em}
\noindent\textbf{Core wall-time quantities.}
\begin{itemize}
  \item \textbf{Timestep wall time} \(t_{\text{step}}\): average wall time per timestep (primary objective).
  \item \textbf{Phase-aware time}: where possible, decompose \(t_{\text{step}}\) into dominant phases (e.g., globally coupled solve phases vs element-local work) to avoid misattributing wins.
\end{itemize}

\vspace{0.35em}
\noindent\textbf{Imbalance and efficiency.}
\begin{itemize}
  \item \textbf{Imbalance ratio} \(\rho=\frac{\max(\text{rank time/load})}{\text{mean}(\text{rank time/load})}\). Ideal \(\rho=1\).
  \item \textbf{Effective efficiency} \(\eta \approx \frac{\text{mean}}{\max}\). Ideal \(\eta \approx 1\).
\end{itemize}

\vspace{0.35em}
\noindent\textbf{Overhead and churn (what must be bounded).}
\begin{itemize}
  \item \textbf{Rebalance events}: number of rebalances executed.
  \item \textbf{No-op events}: rebalances that move 0 elements / negligible state (a direct indicator of wasted periodic scheduling).
  \item \textbf{Migration intensity}: elements moved and migration time; used to enforce “bounded movement”.
\end{itemize}

\vspace{0.35em}
\noindent\textbf{Periodic vs triggered.}
A periodic policy rebalances on a fixed cadence \(k\) and can pay overhead even when the mapping changes weakly.
A triggered policy executes a rebalance only when a model predicts positive net gain, typically framed as \(\Delta T_{\text{benefit}} > \Delta T_{\text{cost}}\), and should reduce unnecessary events while keeping overhead bounded.
