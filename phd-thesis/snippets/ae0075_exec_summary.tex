\section*{Executive Summary}

High-fidelity CFD is increasingly limited not by whether a method is accurate, but by whether it runs \emph{predictably} and \emph{efficiently} at scale.
Adaptive high-order methods are attractive because they concentrate numerical effort where the physics demands it; however, that same adaptivity creates strongly uneven work distributions that can throttle parallel throughput.
This proposal directly addresses the AE0075 studentship topic on novel algorithms for load rebalancing of adaptive simulations on heterogeneous clusters \cite{ImperialAE0075Studentship}.

On modern clusters, two effects collide.
First, the wall time of a timestep is often governed by the \emph{slowest} partition because synchronized phases and globally coupled solves force ranks/devices to wait for one another.
Second, heterogeneous hardware means that “equal work” depends on device throughput and on which phases dominate the timestep.
As a result, simple heuristics (e.g., equal element counts, or rebalancing periodically every \(k\) steps) can waste compute time: they can leave expensive regions co-located on the same partition, or trigger rebalances whose overhead exceeds their benefit.

\vspace{0.35em}
\noindent\textbf{Thesis.}
Achieve predictable throughput for adaptive high-order CFD on heterogeneous systems by making rebalancing a measurable decision:
(i) calibrate models for \emph{benefit} (how much time is saved by improving the mapping) and \emph{cost} (repartition+migration+coordination overhead),
(ii) compute heterogeneity-aware weights/mappings, and
(iii) rebalance only when predicted net gain is positive, under bounded migration to prevent churn.

\vspace{0.35em}
\noindent\textbf{Preliminary signals (measured, motivating the approach).}
\begin{itemize}
  \item Per-element cost rises sharply with local approximation order (polynomial order \(P\), defined in Notation): measured timestep cost scales approximately as \(P^{2.68}\) in a representative sweep (\(r^2 \approx 0.9985\)). So: balance must be cost-weighted, not count-based.
  \item Localized refinement produces time hotspots: clustered refinement yields max/mean load ratios \(\approx 2.7\text{--}4.0\). So: adaptive meshes can throttle wall time if hotspots co-locate on one rank/device.
  \item Not all timestep time is “balance-sensitive”: pressure+viscous phases can consume \(\sim 0.8\text{--}0.85\) of the timestep at higher core counts. So: gains must be evaluated phase-by-phase; some time is intrinsically global-coupled.
  \item Decision policies matter: triggered rebalancing can match periodic performance while reducing rebalance events. So: the primary lever is a reliable gate based on predicted net gain.
\end{itemize}

\noindent\textbf{Contributions.}
\begin{itemize}
  \item A Nektar++-anchored hook-point map for the adapt \(\leftrightarrow\) partition lifecycle.
  \item A calibrated performance/overhead model supporting heterogeneity-aware weighting and phase-aware evaluation.
  \item A triggered, bounded-migration rebalancing policy evaluated against periodic baselines with explicit overhead accounting.
\end{itemize}
