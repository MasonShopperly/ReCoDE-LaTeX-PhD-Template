\section*{Executive Summary}

This proposal addresses Imperial College London’s AE0075 PhD studentship on \emph{novel algorithms for load rebalancing of adaptive simulations on heterogeneous clusters}, using Nektar++ as the implementation and evaluation platform.

High-fidelity CFD is increasingly limited not by whether a method is accurate, but by whether it runs \emph{predictably} and \emph{efficiently} at scale.
Adaptive high-order methods are attractive because they concentrate numerical effort where the physics demands it; however, that same adaptivity can create strongly uneven work distributions that throttle parallel throughput.

Two practical effects drive the problem.
First, many timestep phases include synchronization or globally coupled solves, so end-to-end wall time is governed by the \emph{slowest} rank or device.
Second, heterogeneous hardware (mixed CPU/GPU nodes, differing throughput) makes “equal work” depend on both device capability and which phases dominate the timestep.
As a result, simple heuristics—such as equal element counts, or periodic “rebalance every \(k\)” policies—can waste compute time: expensive regions can be co-located on one partition, and rebalancing can be triggered even when its overhead exceeds its benefit.

In this proposal, \emph{adaptivity} means the simulation changes where resolution is concentrated over time.
For high-order element methods, that resolution is controlled not only by mesh size \(h\) but also by polynomial degree \(P\); when \(P\) varies spatially, equal element counts do not imply equal compute time.
Against this backdrop, the core premise is that rebalancing should be treated as a \emph{measurable decision} with explicit benefit–cost accounting, rather than a fixed periodic schedule.
