\section*{Measured feasibility (what we already know, and what it implies)}
\noindent This section summarizes the current evidence pack (scripts and raw logs are tracked for reproducibility (available on request)) and the \emph{actionable implications} for the research plan.

\vspace{0.4em}
\subsection*{1) Element cost growth with \(P\) is strongly superlinear}
The measured \(P\)-sweep shows per-step cost grows approximately as \(P^\alpha\) with \(\alpha>2\) (e.g., \code{p\_sweep\_re100.csv}).
\textbf{Implication:} “balanced element counts” is not a meaningful balance objective in adaptive spectral/\(hp\): balance must be time/cost-weighted.

\vspace{0.4em}
\subsection*{2) Hotspots create large imbalance even with the same mesh cardinality}
Clustered refinement produces naive max/mean ratios \(\gg 1\) (\code{imbalance\_demo\_clustered.csv}).
\textbf{Implication:} we need an imbalance metric aligned with wall time (rank-local timers or calibrated per-element weights), not purely structural measures.

\vspace{0.4em}
\subsection*{3) Global solver phases dominate (limits ceiling of achievable speedups)}
Timer composition indicates pressure/viscous phases can dominate Execute (\code{cylinder2d\_timer\_profile\_p3p6\_fin20\_v2.csv}).
\textbf{Implication:} rebalancing wins must be evaluated at the \emph{phase level} (what fraction of time is imbalance-sensitive vs globally coupled).

\vspace{0.4em}
\subsection*{4) Triggered policies can reduce wasted/no-op rebalances}
Triggered vs periodic comparisons show the promise of cutting events while preserving outcomes (\code{periodic\_vs\_trigger\_static\_rb85\_mig90\_re100.csv}; \code{key\_results.md}).
\textbf{Implication:} the core research lever is a model that predicts \(\Delta T_{\text{benefit}}\) and \(\Delta T_{\text{cost}}\) tightly enough to act as a gate.
