% -------------------------------------------------------------
% Example: Multi-panel Figures for the Results Chapter
% -------------------------------------------------------------
A dedicated results chapter is a standard component across most disciplines and thesis formats.  
Clear and effective graphical representation is crucial in this section, as figures often communicate trends, comparisons, and patterns more efficiently than text.

% -------------------------------------------------------------
% 2×2 Multi-panel Figure (Four Panels)
% -------------------------------------------------------------
Figure~\ref{fig:four_panels} presents a 2×2 layout with four panels (a–d), which is a common structure when comparing multiple experimental conditions, time points, or parameter variations.  
Each panel can represent a different scenario, and together they provide a cohesive visual summary of the experiment. The figures were obtained at \cite{Gomes2022}.

% START SNIPPET: figures_four
\begin{figure}[h!]
    \centering 

    % -------------------
    % Row 1
    % -------------------
    \begin{subfigure}[b]{0.4\textwidth}
        \centering
        \includegraphics[width=\textwidth]figures/Picture 0.png}
        \caption{Image a}
        \label{fig:sub_a}
    \end{subfigure}%
    \hfill
    \begin{subfigure}[b]{0.4\textwidth}
        \centering
        \includegraphics[width=\textwidth]{figures/Picture 1.png}
        \caption{Image b}
        \label{fig:sub_b}
    \end{subfigure}
    

    % -------------------
    % Row 2
    % -------------------
    \begin{subfigure}[b]{0.4\textwidth}
        \centering
        \includegraphics[width=\textwidth]{figures/Picture 2.png}
        \caption{Image c}
        \label{fig:sub_c}
    \end{subfigure}%
    \hfill
    \begin{subfigure}[b]{0.4\textwidth}
        \centering
        \includegraphics[width=\textwidth]{figures/Picture 5.png}
        \caption{Image d}
        \label{fig:sub_d}
    \end{subfigure}

    % -------------------
    % Overall caption 
    % -------------------
    \caption[Four-panel figure]{Four-panel figure showing different aspects of the experiment.  
    Subfigures a–d correspond to distinct experimental conditions or processing steps.}
    \label{fig:four_panels}
\end{figure}
% END SNIPPET: figures_four

% Notes for students:
% - Adjust subfigure widths (0.4\textwidth) to control spacing.
% - Replace images with your own file paths.
% - Optional: Use \caption[] for a shorter List of Figures entry.

% -------------------------------------------------------------
% Two-panel Figure (Side-by-side)
% -------------------------------------------------------------
When comparing two related outcomes, measurements, or conditions, 
a two-panel figure is often the most effective option.  
Figure~\ref{fig:two_panels} presents panels a) and b) side by side, 
allowing easy visual comparison between the two cases.
% START SNIPPET: figures_two
\begin{figure}[h!]
    \centering

    % Panel a (left)
    \begin{subfigure}[b]{0.48\textwidth}
        \centering
        \includegraphics[width=\textwidth]{figures/Picture 5.png}
        \caption{Description of image a}
        \label{fig:sub_a1}
    \end{subfigure}
    \hfill
    % Panel b (right)
    \begin{subfigure}[b]{0.48\textwidth}
        \centering
        \includegraphics[width=\textwidth]{figures/Picture 2.png}
        \caption{Description of image b}
        \label{fig:sub_b1}
    \end{subfigure}

    \caption[Two-panel figure]{Two-panel figure comparing condition a) and condition b).}
    \label{fig:two_panels}
\end{figure}
% END SNIPPET: figures_two

% Notes for students:
% - Use 0.48\textwidth for tight spacing across the full width.
% - Adjust file names, captions, and labels as needed.
% - You can reference subfigures using \ref{fig:sub_a1}, etc.
