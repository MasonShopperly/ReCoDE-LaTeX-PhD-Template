In many experiments, it is useful to present multiple related images together to illustrate comparisons, trends, or different conditions. 
Figure~\ref{fig:four_panels} shows a 2×2 layout with four panels (a–d), which allows the reader to quickly compare results across different experimental settings or data processing steps. 
Each panel can represent a separate condition, time point, or parameter value, while the overall figure provides a cohesive summary.


\begin{figure}[h!]
    \centering
    % Row 1
    \begin{subfigure}[b]{0.4\textwidth}
        \centering
        \includegraphics[width=\textwidth]{phd-thesis/figures/Picture 0.png}
        \caption{Image a}
        \label{fig:sub_a}
    \end{subfigure}%
    \hfill
    \begin{subfigure}[b]{0.4\textwidth}
        \centering
        \includegraphics[width=\textwidth]{phd-thesis/figures/Picture 1.png}
        \caption{Image b}
        \label{fig:sub_b}
    \end{subfigure}
    
    % Row 2
    \begin{subfigure}[b]{0.4\textwidth}
        \centering
        \includegraphics[width=\textwidth]{phd-thesis/figures/Picture 2.png}
        \caption{Image c}
        \label{fig:sub_c}
    \end{subfigure}%
    \hfill
    \begin{subfigure}[b]{0.4\textwidth}
        \centering
        \includegraphics[width=\textwidth]{phd-thesis/figures/Picture 5.png}
        \caption{Image d}
        \label{fig:sub_d}
    \end{subfigure}
    
    % Overall caption
    \caption[Four-panel figure]{Four-panel figure showing different aspects of the experiment. Subfigures a)–d) correspond to different conditions.}
    \label{fig:four_panels}
\end{figure}

When comparing two related results or experimental conditions, it is often convenient to present them side by side in a two-panel figure. 
Figure~\ref{fig:two_panels} shows panels a) and b), which allow the reader to easily observe differences or trends between the two conditions. 
Each panel can represent a distinct variable, time point, or treatment, while the overall figure provides a cohesive comparison.


\begin{figure}[h!]
    \centering
    % Subfigure a
    \begin{subfigure}[b]{0.48\textwidth}
        \centering
        \includegraphics[width=\textwidth]{phd-thesis/figures/Picture 5.png}
        \caption{Description of image a}
        \label{fig:sub_a}
    \end{subfigure}%
    \hfill
    % Subfigure b
    \begin{subfigure}[b]{0.48\textwidth}
        \centering
        \includegraphics[width=\textwidth]{phd-thesis/figures/Picture 2.png}
        \caption{Description of image b}
        \label{fig:sub_b}
    \end{subfigure}
    
    % Overall caption
    \caption[Two-panel figure]{Two-panel figure showing the comparison between condition a) and condition b).}
    \label{fig:two_panels}
\end{figure}
