\providecommand{\code}[1]{\texttt{#1}}
\makeatletter
\@ifundefined{path}{\providecommand{\codepath}[1]{\texttt{#1}}}{\providecommand{\codepath}[1]{\path{#1}}}
\makeatother
\markboth{}{}

% detokenize removed
\providecommand{\AEQCoverMode}{0}
% AE0075 Stage-2 Proposal (Draft v2)
% Contract: No invented achievements. No invented quotes. Use labels EXACTLY:
% VERIFIED / UNKNOWN / UNVERIFIED / ASSUMED

% AE0075_ONLY: suppressed mini-cover heading

% 
% AE0075_ONLY: suppressed mini-title
%\section*{AE0075 Stage-2 Research Proposal (Draft)}

% (counter suppressed)% (counter suppressed)% (toc entry suppressed)
% ===================== SUBMISSION-POLISH PATCH BEGIN =====================
% Identity / metadata
\ifnum\AEQCoverMode=0\relax
\begin{center}
{\Large \textbf{AE0075 Stage-2 Research Proposal}}\\[0.4em]
{\normalsize \textit{Studentship title: TBD (paste verbatim from Imperial listing)}}\\[0.8em]
{\large \textbf{Mason Shopperly}}\\
{\normalsize \today}
\end{center}
\fi

\vspace{0.5em}

\section*{Executive Summary}

High-fidelity CFD is increasingly limited not by whether a method is accurate, but by whether it runs \emph{predictably} and \emph{efficiently} at scale.
Adaptive high-order methods are attractive because they concentrate numerical effort where the physics demands it; however, that same adaptivity creates strongly uneven work distributions that can throttle parallel throughput.

On modern clusters, two effects collide.
First, the wall time of a timestep is often governed by the \emph{slowest} partition because synchronized phases and globally coupled solves force ranks/devices to wait for one another.
Second, heterogeneous hardware means that “equal work” depends on device throughput and on which phases dominate the timestep.
As a result, simple heuristics (e.g., equal element counts, or rebalancing periodically every \(k\) steps) can waste compute time: they can leave expensive regions co-located on the same partition, or trigger rebalances whose overhead exceeds their benefit.

\vspace{0.35em}
\noindent\textbf{Thesis.}
Achieve predictable throughput for adaptive high-order CFD on heterogeneous systems by making rebalancing a measurable decision:
(i) calibrate models for \emph{benefit} (how much time is saved by improving the mapping) and \emph{cost} (repartition+migration+coordination overhead),
(ii) compute heterogeneity-aware weights/mappings, and
(iii) rebalance only when predicted net gain is positive, under bounded migration to prevent churn.

\vspace{0.35em}
\noindent\textbf{Preliminary signals (measured, motivating the approach).}
\begin{itemize}
  \item \emph{Per-element cost rises sharply with local approximation order} (polynomial order \(P\)): measured timestep cost scales approximately as \(P^{2.68}\) in a representative sweep (\(r^2 \approx 0.9985\)). \textbf{So:} balance must be cost-weighted, not count-based.
  \item \emph{Localized refinement produces time hotspots:} clustered refinement yields max/mean load ratios \(\approx 2.7\text{--}4.0\). \textbf{So:} adaptive meshes can throttle wall time if hotspots co-locate on one rank/device.
  \item \emph{Not all timestep time is “balance-sensitive”:} pressure+viscous phases can consume \(\sim 0.8\text{--}0.85\) of the timestep at higher core counts. \textbf{So:} gains must be evaluated phase-by-phase; some time is intrinsically global-coupled.
  \item \emph{Decision policies matter:} triggered rebalancing can match periodic performance while reducing rebalance events. \textbf{So:} the primary lever is a reliable gate based on predicted net gain.
\end{itemize}

\vspace{0.35em}
\noindent\textbf{Contributions.}
\begin{itemize}
  \item A Nektar++-anchored hook-point map for the adapt \(\leftrightarrow\) partition lifecycle.
  \item A calibrated performance/overhead model supporting heterogeneity-aware weighting and phase-aware evaluation.
  \item A triggered, bounded-migration rebalancing policy evaluated against periodic baselines with explicit overhead accounting.
\end{itemize}

\section*{Introduction, problem formulation, and positioning}

Adaptive spectral/\(hp\) CFD concentrates resolution where physics demands it, which is why it is attractive for high-fidelity simulations on practical meshes.
The same mechanism that makes it powerful also makes it difficult to run efficiently at scale: adaptation changes the \emph{shape} of the workload over time.
Regions of the domain may carry significantly higher polynomial order \(P\) (and/or smaller element size \(h\)), and those localized regions can dominate timestep cost if they accumulate on the same rank or device.

At scale, end-to-end wall time is frequently set by the slowest partition.
This is not merely an implementation detail: globally coupled phases (in particular solver phases) introduce synchronization that forces ranks/devices to wait for one another.
Heterogeneous clusters amplify the effect further—mixed CPUs, SMT, and GPUs mean that “equal work” depends on device throughput and on which phases dominate the timestep.
Consequently, balancing by element count is not meaningful in adaptive high-order settings: equal counts do not imply equal time when per-element cost varies strongly with \(P\).

The decision problem addressed here is therefore operational and model-based:
given an adaptive simulation whose \(P\)-distribution evolves, \emph{when} should a repartition/migration be performed, and \emph{how much} data should be moved, to reduce wall time without creating churn?
A periodic policy (“rebalance every \(k\)”) is simple but can pay overhead even when the mapping changes weakly, including no-op or near-no-op events.
A triggered policy can be better, but only if it can predict net gain reliably enough to act as a gate and if it controls migration so that rebalancing does not become the dominant cost.

We evaluate policies using phase-aware wall-time metrics and explicit overhead accounting: wall-time per step \(t_{\text{step}}\), imbalance ratio \(\rho=\max/\text{mean}\), effective efficiency \(\eta\approx \text{mean}/\max\), event counts (including no-ops), and migration intensity (elements moved and migration time).
A policy is “better” only if it reduces end-to-end wall time while keeping overhead bounded and preserving solution acceptability (QoI tolerance).

This proposal is grounded in the spectral element and spectral/\(hp\) CFD foundations \cite{Patera1984SpectralElement,KarniadakisSherwin2005}, and it is anchored in Nektar++ as a credible target platform for publishable algorithm work \cite{Cantwell2015NektarPP,Cantwell2020NektarPP}.
On the load-balancing side, partitioning and dynamic rebalancing baselines are well-established in HPC (e.g., multilevel graph partitioning \cite{KarypisKumar1998METIS} and dynamic services/frameworks \cite{Devine2002Zoltan}), and scalable adaptive-mesh infrastructures motivate what “practical overhead” looks like in real systems \cite{Burstedde2011p4est}.
The gap we focus on is the intersection of (i) superlinear cost growth with \(P\), (ii) explicit modeling of rebalance overhead, and (iii) heterogeneity-aware decision rules that avoid wasted events and migration churn.

The remainder of this proposal makes that gap precise: it specifies a model for predicted benefit versus cost, defines a trigger rule with bounded migration, and lays out an evaluation plan designed to produce phase-aware, publishable performance evidence.

\section*{Research aims, hypotheses, and evaluation logic}

This project treats load rebalancing as a measurable decision within adaptive high-order simulation on heterogeneous clusters: given that the cost distribution changes over time, decide \emph{when} to rebalance and \emph{how much} to migrate so that timestep wall time decreases without creating migration churn.

Rather than a long list of features, the work is organized around three concrete questions:
\begin{enumerate}
  \item \textbf{Savings:} if we change the mapping, how much wall time is saved (phase-aware)?
  \item \textbf{Cost:} what overhead do we pay to repartition, migrate, and coordinate?
  \item \textbf{Decision:} under what conditions is a rebalance net-positive, and how do we bound movement?
\end{enumerate}

These questions become falsifiable hypotheses:
\begin{itemize}
  \item \textbf{H1 (savings is predictable):} a cost-weighted mapping calibrated to order sensitivity reduces slowest-part throttling compared to count-based baselines, especially under clustered refinement.
  \item \textbf{H2 (cost matters):} in realistic adaptive cadence, periodic policies waste time in regimes where mappings change weakly (including no-op / near-no-op events) because overhead exceeds benefit.
  \item \textbf{H3 (a gate beats a schedule):} a triggered rule that rebalances only when \(\Delta T_{\text{benefit}}>\Delta T_{\text{cost}}\), combined with a migration budget, matches or improves wall time versus periodic rebalancing while reducing event count and preventing churn.
\end{itemize}

\noindent\textbf{Evaluation logic.}
We compare against simple baselines (static count-based, static cost-weighted, periodic cadence) across controlled regimes (hotspot severity, adaptive cadence, heterogeneity).
A result counts only if it reduces \(t_{\text{step}}\) while keeping overhead bounded, limiting migration volume, and maintaining solution acceptability (QoI tolerance).


% ====================== SUBMISSION-POLISH PATCH END ======================

\label{ch:ae0075}
% (toc entry suppressed)

% --- labels (EXACT strings) ---
\newcommand{\VERIFIED}{\textbf{VERIFIED}}
\newcommand{\UNKNOWN}{\textbf{UNKNOWN}}
\newcommand{\UNVERIFIED}{\textbf{UNVERIFIED}}
\newcommand{\ASSUMED}{\textbf{ASSUMED}}

% --- claim/evidence hooks ---
\newcommand{\Claim}[1]{\textbf{[Claim #1]}}
\newcommand{\Evidence}[1]{\textit{Evidence: #1}}
\newcommand{\Reproduce}[1]{\texttt{Reproduce: #1}}

% Breakable monospace paths (prevents overfull hbox spam)
% DUPLICATE codepath removed
% Keep ugly draft readable while we iterate
\begingroup
\sloppy

\endgroup

% === AE0075 WORLD-CLASS ADDENDUM START ===
\clearpage

% === AE0075_PRELIM_RESULTS_START ===

\section{Preliminary results and evidence base}
\label{sec:ae0075-prelim}



\noindent This section summarizes what is already quantified in the current Nektar++ Cylinder2D runs, and the specific measurements we add next to extend the same analysis to MPI, heterogeneity, and a QoI tolerance.
\vspace{0.4em}
\noindent\textit{Measured anchor (Cylinder2D).} Timestep cost grows strongly and superlinearly with order. A compact fit captures the observed scaling:
\[
\text{ms/step} \approx k\,P^{a}.
\]
The current calibrated parameters are:

\begin{center}
\begin{tabular}{lccc}
\hline
Dataset & $k$ & $a$ & $r^2$ \\
\hline
Re=100 (primary) & 0.223522 & 2.682 & 0.9985 \\
Re=40 (secondary) & 0.315049 & 2.467 & 0.9962 \\
\hline
\end{tabular}
\end{center}

\noindent This is the starting point for time-based weights when \(P\) varies in space: equal element counts do not imply equal time.

\vspace{0.4em}
\noindent\textit{Grounded offline signal (hotspot imbalance).} Using the measured fit above as element weights, clustered high-\(P\) regions create concentrated cost that a count-based partition can co-locate. In a clustered ``wake-like'' ordering with contiguous partitions, the resulting naive max/mean imbalance ratios are on the order of $\approx 2.7$--$4.0$ (offline demo grounded by the measured fit).

\vspace{0.4em}
\noindent\textit{Grounded offline signal (policy behavior).} In regime sweeps with explicit migration penalty, fixed-cadence schedules can spend rebalance events on no-op or near-no-op remaps (0 elements moved), and that wasted fraction is cadence-sensitive. A triggered policy removes no-op events across sweeps and can match periodic performance with fewer rebalance calls in near-crossover regimes (e.g., \(40\rightarrow 18\) events).

\vspace{0.4em}
\noindent\textit{What we measure next (commitments).} The next runs add the missing decision terms directly: (i) per-rank (and device) time distributions in MPI to confirm slowest-partition throttling, (ii) overhead decomposition for repartition+migration+rebuild, and (iii) a QoI definition with a tolerance so ``faster'' remains valid.

% === AE0075_PRELIM_RESULTS_END ===





\section{Execution credibility (measured evidence + reproducible artifacts)}

\textbf{Evidence discipline.} Every non-trivial performance claim below corresponds to a tracked artifact under \texttt{supplementary material (available on request)} (or is explicitly marked \textbf{ASSUMED} / \textbf{UNVERIFIED} pending verification). This proposal is intentionally ``measurement-first'' and self-policing.

\vspace{0.5em}
\noindent\textbf{What is already measured (and re-runnable):}
\begin{itemize}
 \item \textbf{Measured p-scaling fit (real Nektar++ Cylinder2D):} ms/step $\approx k \cdot P^{a}$ with $(k,a,r^2)$ recorded for Re=100 primary. Reproduce: \texttt{bash runs/cylinder2d/p\_sweep.sh}. Output: \texttt{p\_sweep\_re100.csv}.
 \item \textbf{Trigger vs periodic:} event-driven trigger removes wasted no-op rebalances and can match periodic performance with fewer events (e.g., 40 $\rightarrow$ 18 rebalances in a near-crossover regime). Backed by tracked comparison CSVs and summarized in \texttt{key\_results.md}.
 \item \textbf{Timer composition:} global solves dominate Execute time, motivating solve-aware rebalancing rather than naive element-count balancing.
 \item \textbf{Heterogeneity sensitivity:} effective efficiency can collapse sharply under controlled slowdowns, motivating heterogeneity-aware weighting and bounded migration.
\end{itemize}

% AE0075_EVAL_TABLE_V2_START
\begin{table}[H]
\centering
\small
\setlength{\tabcolsep}{5pt}
\renewcommand{\arraystretch}{1.18}
% AE0075_EVAL_MATRIX_V2_START
\begin{tabular}{p{0.26\linewidth} p{0.26\linewidth} p{0.33\linewidth} p{0.10\linewidth}}
\toprule
\textbf{Question} & \textbf{Metric(s)} & \textbf{Evidence (tracked)} & \textbf{Status} \\
\midrule
P-cost superlinear in \(P\)? &
fit params \(k,a,r^2\) &
\codepath{p\_sweep\_re100.csv} (fit via \codepath{runs/cylinder2d/pack\_tables.py}) &
\textbf{MEASURED} \\
Clustered hp hotspots cause imbalance? &
max/mean, mean/max &
\codepath{imbalance\_demo\_clustered.csv} (grounded in measured fit) &
\textbf{MODELED} \\
Global solves dominate at scale? &
(Pressure+Viscous)/Execute &
\codepath{cylinder2d\_timer\_profile\_p3p6\_fin20\_v2.csv} &
\textbf{MEASURED} \\
Heterogeneity collapses effective efficiency? &
effective efficiency vs baseline &
\codepath{cylinder2d\_mpi\_hetero\_amp\_p6\_fin20\_r3\_summary.csv} &
\textbf{MEASURED} \\
Trigger reduces no-op rebalances? &
events, moved\_elems\_total, avg\_ms &
\codepath{rebalance\_trade\_regime\_*cmp\_trigger*\_re100.csv} &
\textbf{MODELED} \\
QoI invariant under rebalance? &
QoI drift tolerance &
\textit{Define QoI + measure on at least one adaptive case (next).} &
\textbf{ASSUMED} \\
\bottomrule
\end{tabular}
% AE0075_EVAL_MATRIX_V2_END

\caption{Evaluation matrix (clean layout; targets are fixed once portal constraints and QoI are defined).}
\end{table}
% AE0075_EVAL_TABLE_V2_END

\clearpage
% AE0075_TECH_APPROACH_START
\clearpage
\section{Technical Approach: Triggered Rebalancing with Bounded Migration}

\noindent\textbf{Objective.} Replace periodic ``rebalance every $k$'' with a \emph{cost--benefit trigger} and a \emph{migration budget} so rebalancing happens only when it is predicted to reduce time-to-solution more than it costs (coordination + data movement), and never induces unstable churn.

\subsection*{A. Performance model (measured, then used for decisions)}
We decompose per-step time as
\[
T_{\text{step}} \;\approx\; T_{\text{compute}}(P,\text{elem}) \;+\; T_{\text{comm}}(\partial\Omega,\text{messages}) \;+\; T_{\text{solve}}(\text{global}).
\]
\begin{itemize}
 \item \textbf{Compute:} use a calibrated per-element cost model, with a strong prior that cost grows superlinearly with local polynomial order \(P\). (Calibration inputs: p-sweep CSVs.)
 \item \textbf{Communication:} model as a function of cut edges / message counts / bytes; keep it simple initially (linear fit) and refine only if it materially changes decisions.
 \item \textbf{Global solve sensitivity:} incorporate the measured dominance of global solves at scale as an explicit term so that rebalancing decisions are not blind to the true bottleneck.
\end{itemize}

\subsection*{B. Imbalance and predicted benefit}
Let \(t_r\) be the modeled per-step time contribution for rank (or device) \(r\). Define an imbalance ratio
\[
\rho \;=\; \frac{\max_r t_r}{\mathrm{mean}_r\, t_r},
\qquad \text{and an efficiency proxy}\qquad
\eta \approx \frac{1}{\rho}.
\]
A candidate remap \(M\rightarrow M'\) yields a predicted improvement
\[
\Delta T_{\text{benefit}} \;\approx\; T_{\text{step}}(M)\;-\;T_{\text{step}}(M').
\]
\textbf{Key point:} \(\Delta T_{\text{benefit}}\) is computed from the calibrated model, so it can be evaluated cheaply and repeatedly across adaptation steps.

\subsection*{C. Trigger rule (rebalance only when it is worth it)}
We rebalance when predicted savings exceed predicted overhead:
\[
\textbf{Trigger if}\quad \Delta T_{\text{benefit}} \;>\; \gamma\, T_{\text{overhead}},
\]
where \(T_{\text{overhead}} = T_{\text{coord}} + T_{\text{migrate}} + T_{\text{rebuild}}\) and \(\gamma\ge 1\) is a safety margin (tuned by robustness sweeps).
\begin{itemize}
 \item \textbf{Coordination:} barrier + partition computation + metadata exchange.
 \item \textbf{Migration:} element/DOF transfer + field remap cost (see bounded migration below).
 \item \textbf{Rebuild:} communication pattern rebuild and any solver-side reinitialization.
\end{itemize}

\subsection*{D. Bounded migration (prevent churn; match heterogeneous constraints)}
A pure trigger can still choose aggressive remaps. We enforce a budget:
\[
\textbf{moved\_cost}(M\rightarrow M') \;\le\; B,
\]
where \(\textbf{moved\_cost}\) is \emph{not} just ``moved elements''. A stronger proxy (target for implementation) is \textbf{P-weighted DOFs moved} (or an equivalent \(hp\)-aware measure). This aligns the migration penalty with what actually dominates runtime and bandwidth.

\subsection*{E. Algorithm sketch (decision loop)}
\begin{enumerate}
 \item After each adapt step, compute per-rank modeled costs \(t_r\) and current imbalance \(\rho\).
 \item Propose a candidate remap \(M'\) using heterogeneity-aware weights (WP2) under budget \(B\).
 \item Compute \(\Delta T_{\text{benefit}}\) and \(T_{\text{overhead}}\).
 \item If \(\Delta T_{\text{benefit}} > \gamma T_{\text{overhead}}\), apply remap; else skip.
 \item Log: predicted vs realized \(T_{\text{step}}\), overhead, and any mismatch to refine the model.
\end{enumerate}

\subsection*{F. What this achieves (why it solves more than “it’s a problem”)}
\begin{itemize}
 \item \textbf{No-op elimination:} if the remap is predicted not to help, it is not executed.
 \item \textbf{Overhead control:} the budget \(B\) bounds worst-case data movement and rebuild churn.
 \item \textbf{Heterogeneity awareness:} weights encode device/rank speed so ``balanced'' means balanced in \emph{time}, not element count.
 \item \textbf{Measurable progress:} each component (model fit, trigger threshold, migration budget) is validated via tracked CSV sweeps and regenerated tables/plots.
\end{itemize}

\noindent\textbf{Reproduce (UNVERIFIED path until wired):} add a single driver script that (i) runs a param sweep over \(\gamma\) and \(B\), (ii) emits tracked CSVs, and (iii) regenerates the proposal tables/figures from those CSVs.
% AE0075_TECH_APPROACH_END

\section{Research plan (work packages + crisp deliverables)}

\subsection*{WP0 --- Evidence harness and reproducibility (continuous)}
\begin{itemize}
 \item Maintain one-command doc rebuild and keep all headline numbers generated from tracked CSVs.
 \item Deliverable: \texttt{key\_results.md} remains numbers-first source of truth.
\end{itemize}

\subsection*{WP1 --- Nektar++ hook points (feasibility mapping; Gate-2)}
\begin{itemize}
 \item Identify where element work / polynomial order / timers expose imbalance signals.
 \item Identify where repartition + migration can be inserted in an adaptive loop.
 \item Deliverable: \texttt{NEKTARPP\_HOOKS.md} with file paths + entry points (\textbf{UNVERIFIED} until Nektar++ source is in-scope here).
\end{itemize}

\subsection*{WP2 --- Static heterogeneity-aware weighting}
\begin{itemize}
 \item Extend weights to account for rank-speed model + solver-phase dominance (solve-aware weighting).
 \item Evaluate on controlled slowdown perturbations and report net wall-time and efficiency.
\end{itemize}

\subsection*{WP3 --- Dynamic rebalancing (trigger + bounded migration)}
\begin{itemize}
 \item Trigger policy: rebalance only when predicted loss exceeds threshold (no wasted events).
 \item Bounded migration: cap moved units per adapt step; study crossover vs heterogeneity and hotspot severity.
 \item Deliverable: stability of throughput under adaptation with explicitly bounded overhead.
\end{itemize}

\clearpage
\section{Timeline (36 months, with gates and “done means”)}

\begin{table}[H]
\centering
\small
\begin{tabular}{p{0.18\linewidth} p{0.76\linewidth}}
\toprule
Window & Outcomes (each line ends in a concrete artifact) \\
\midrule
Months 1--3 &
WP0 finalize + WP1 hook mapping; publish \texttt{NEKTARPP\_HOOKS.md} (\textbf{UNVERIFIED} until source access),
lock portal constraints, lock QoI definition, freeze metrics. \\
Months 4--9 &
WP2 static heterogeneity-aware weighting; at least one controlled-perturbation win with tracked CSV + regenerated docs. \\
Months 10--18 &
WP3 trigger + bounded migration; regime maps across heterogeneity/hotspot/adapt cadence; “no-op eliminated” robustness tables. \\
Months 19--30 &
Generalize to additional adaptive cases; refine models; draft paper-quality evaluation. \\
Months 31--36 &
Thesis write-up + final validation + dissemination. \\
\bottomrule
\end{tabular}
\caption{High-level timeline (kept honest by artifacts and gates).}
\end{table}

\clearpage
\section*{References}
\printbibliography[heading=none]


% === AE0075 WORLD-CLASS ADDENDUM END ===
