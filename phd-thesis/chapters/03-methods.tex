% -------------------------------------------------------------
% Example: Writing Mathematical Equations in LaTeX
% -------------------------------------------------------------
In many experimental analyses, it is important to quantify the variability present in the measurements.  
In this section, we demonstrate how to typeset equations in \LaTeX{}, using the sample standard deviation as a simple example.

The standard deviation provides a measure of how spread out the data are from the mean.  
For a sample of $n$ observations, $x_1, x_2, \dots, x_n$, the sample standard deviation $s$ is defined as:
% START SNIPPET: equations_std
\begin{equation}
    s = \sqrt{\frac{1}{n-1} \sum_{i=1}^{n} \left(x_i - \bar{x}\right)^2 }
    \label{eq:std_sample}
\end{equation}

% Explanation:
% - The equation environment automatically numbers the equation.
% - \sum creates the summation symbol.
% - \bar{x} denotes the sample mean.
% - Use \label to refer to the equation later in the text.
% END SNIPPET: equations_std

where $\bar{x}$ is the sample mean, given by:
% START SNIPPET: equations_mean
\begin{equation}
    \bar{x} = \frac{1}{n} \sum_{i=1}^{n} x_i
    \label{eq:mean_sample}
\end{equation}
% END SNIPPET: equations_mean
% Tip:
% If you want the equation unnumbered, use \[
%   ... equation ...
% \] instead of \begin{equation}.

Equation~\ref{eq:std_sample} shows that the standard deviation increases as the data points deviate further from the sample mean.  
This form is commonly used when analysing repeated measurements, sensor data, or experimental uncertainty.

