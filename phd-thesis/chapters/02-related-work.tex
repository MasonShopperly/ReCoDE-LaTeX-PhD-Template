% -------------------------------------------------------------
% Example Tables for the Related Work Chapter
% -------------------------------------------------------------
The Related Work chapter is a crucial component of most theses.  
A practical way to organise your project—and to simplify writing, editing, and version control—is to create a separate \LaTeX{} \texttt{.tex} file for each chapter.  
In this section, we demonstrate how to include and format tables within your thesis, using common academic layouts.
Tables can be a valuable tool in your thesis for a variety of purposes, from presenting results to explaining parameters and organizing complex data.

% -------------------------------------------------------------
% 3×3 Example Table
% -------------------------------------------------------------
As shown in Table~\ref{tab:example_3x3}, the following 3×3 table presents placeholder data for parameters X, Y, and Z.  
This structure is useful for showing compact relationships or comparing small sets of variables.
% START SNIPPET: tables_3x3
\begin{table}[h!]
    \centering
    % The optional caption [short title] defines what appears in the List of Tables.
    \caption[3×3 Example Table]{A 3×3 table template with parameters X, Y, and Z.}
    \label{tab:example_3x3}
    \begin{tabular}{l c r} 
        % l = left aligned, c = centred, r = right aligned
        \toprule
        Parameter & Value & Units \\
        \midrule
        X1 & 10 & m \\
        X2 & 20 & m \\
        X3 & 30 & m \\
        \bottomrule
    \end{tabular}
\end{table}
% END SNIPPET: tables_3x3
% -------------------------------------------------------------
% 5-column Example Table
% -------------------------------------------------------------
Table~\ref{tab:example_5col} shows a 5-column table, suitable for experimental outputs, benchmarking studies, or design specifications.  
You can easily extend the number of columns or replace placeholder values with your own data.
% START SNIPPET: tables_5col
\begin{table}[h!]
    \centering
    \caption[5-column Example Table]{A 5-column table template suitable for experimental results or design specifications with parameters A–E.}
    \label{tab:example_5col}
    \begin{tabular}{l c c c c r}
        % lccccr layout:
        % l = sample name left aligned
        % c = four columns centred
        % r = final column right-aligned (useful for numerical data)
        \toprule
        Sample & A & B & C & D & E \\
        \midrule
        1 & -- & -- & -- & -- & -- \\
        2 & -- & -- & -- & -- & -- \\
        3 & -- & -- & -- & -- & -- \\
        \bottomrule
    \end{tabular}
\end{table}
% END SNIPPET: tables_5col
