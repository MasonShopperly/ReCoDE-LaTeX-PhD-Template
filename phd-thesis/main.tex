% START SNIPPET: documentclass
\documentclass[11pt, twoside, openany]{book} 
% 'book' provides chapters, front matter, appendices and a well-structured layout suitable for theses.
% 
% '11pt' sets the base font size. You may select 10pt or 12pt if desired, but 11pt offers excellent readability.
%
% 'twoside' produces a double-sided layout (odd/even pages).
% IMPORTANT: Checklist Section 4.3 requires *centred* text with symmetrical margins—handled in the preamble.
%
% 'openany' allows chapters to begin on either left or right pages.
% For a more traditional thesis layout, use 'openright' (chapters always begin on odd-numbered pages).
%
% Full details about margins, fonts, and styling are explained in the PREAMBLE.
% END SNIPPET: documentclass

% START SNIPPET: preamble_title
% --------------------------------------------------------------------
% PREAMBLE — Global packages and formatting rules
% This section configures fonts, margins, maths, figures, tables,
% code listings, hyperlinks, bibliography formatting, and more.
% It ensures compliance with Imperial College Thesis Requirements
% (Sections 4.1–4.6). Students may add extra packages if needed.
% --------------------------------------------------------------------
% START SNIPPET: preamble
% ------------------------------------------------------
% Encoding and Fonts
% ------------------------------------------------------
\usepackage[utf8]{inputenc}         % Allows UTF-8 characters (e.g., é, ñ, ü).
                                    % Recommended: ensures your thesis handles
                                    % international characters correctly.

\usepackage[T1]{fontenc}            % Improves hyphenation and PDF text extraction.
                                    % Good practice for long academic documents.

\usepackage{lmodern}                % Modern, scalable Latin font family.
                                    % Clean and readable for printed theses.

\usepackage{microtype}              % Improves spacing between letters/words.
                                    % Optional but highly recommended for quality typography.

% ------------------------------------------------------
% Mathematics
% ------------------------------------------------------
\usepackage{amsmath,amssymb}        % Essential math packages for equations, symbols,
                                    % matrices, aligned environments, etc.

% ------------------------------------------------------
% Figures and Captions
% ------------------------------------------------------
\usepackage{graphicx}               % Required to insert images using \includegraphics.
                                    % To scale images: use width=0.5\textwidth, height=..., etc.

\usepackage{caption}                % Controls caption style/spacing.
                                    % To modify caption size: \captionsetup{font=small}

\usepackage{subcaption}             % Allows subfigures (a), (b), etc.
                                    % Useful for 2×2 or side-by-side image layouts.

% ------------------------------------------------------
% Links and PDF Navigation
% ------------------------------------------------------
\usepackage{hyperref}               % Adds clickable links in the PDF.
                                    % Automatically links citations, TOC entries, figures.

\usepackage{bookmark}               % Improves PDF bookmarks stability.
                                    % Best practice for long theses.

% ------------------------------------------------------
% Tables
% ------------------------------------------------------
\usepackage{booktabs}               % High-quality table rules: \toprule, \midrule, \bottomrule.
                                    % For professional-looking tables.

% ------------------------------------------------------
% Acronyms and Nomenclature
% ------------------------------------------------------
\usepackage{acronym}                % For lists of acronyms (optional).

\usepackage{nomencl}                % Creates a nomenclature (list of symbols).
\makenomenclature                   % Generates the nomenclature file.
                                    % Run "makeindex thesis.nlo -s nomencl.ist -o thesis.nls"
                                    % or use Overleaf's built-in tool.

% ------------------------------------------------------
% Page Layout (Margins)
% ------------------------------------------------------
\usepackage{geometry}               
\geometry{
  a4paper,
  inner=3cm, outer=3cm,             % Left/right margins — MUST be symmetrical 
                                    % to satisfy Checklist Section 4.3.
  top=3cm, bottom=3cm
}
                                    % To change margins: modify these values.
                                    % Keep left = right to maintain centred page content.

% ------------------------------------------------------
% Code Listings
% ------------------------------------------------------
\usepackage{minted}                 % For syntax-highlighted code blocks (Python, C++, etc.).
                                    % Usage: \begin{minted}{python} ... \end{minted}
                                    %
                                    % IMPORTANT:
                                    % Overleaf → Menu → Compiler → Enable “Shell escape”.
                                    % Without shell-escape, minted will not compile.

% ------------------------------------------------------
% Bibliography (BibLaTeX + Biber)
% ------------------------------------------------------
\usepackage[
  backend=biber,                    % Biber handles UTF-8, DOIs, URLs, many authors, etc.
  style=authoryear,                 % Produces citations like Smith (2020).
  maxcitenames=2                    % Use “et al.” after 2 authors.
]{biblatex}

\addbibresource{bibliography.bib}   % Main bibliography file.
                                    % Export references from Mendeley, Zotero, etc.
                                    %
                                    % Imperial does NOT mandate a specific citation style.
                                    % Any consistent academic style is acceptable.

% ------------------------------------------------------
% Section Heading Style
% ------------------------------------------------------
\usepackage{sectsty}
\allsectionsfont{\normalfont\scshape}
                                    % Converts section titles to small caps.
                                    % Optional: change to \bfseries or remove styling entirely.

% ------------------------------------------------------
% Custom Commands
% ------------------------------------------------------
\newcommand{\degree}{Ph\ensuremath{.\!}D.\xspace}
                                    % Convenience macro for correctly typesetting "Ph.D."
                                    % You can define similar commands for common terms.
% END SNIPPET: preamble

  % Loads all global formatting, packages, and margin settings.

% -------------------------------
% Header and Footer Style
% -------------------------------
\usepackage{fancyhdr}  
% The 'fancyhdr' package gives full control over headers and footers.
% Students can customise:
%   - page numbers (left/centre/right)
%   - add chapter titles in the header
%   - add rules/lines or remove them
%   - different layouts for odd/even pages when using 'twoside'

\pagestyle{fancy}   % Apply the fancy header/footer style.
\fancyhf{}          % Clear all header and footer fields.
\fancyfoot[C]{\thepage}   % Page number centred at the bottom (Checklist Section 4.4).

\renewcommand{\headrulewidth}{0pt}  % Remove top horizontal line.
\renewcommand{\footrulewidth}{0pt}  % Remove bottom horizontal line.

% -------------------------------
% START OF DOCUMENT
% -------------------------------
\begin{document}

% -------------------------------
% Title Page
% -------------------------------
\pagenumbering{arabic}  % Imperial requires Arabic page numbers starting on the title page.
\setcounter{page}{1}    % Title page = Page 1
\begin{titlepage}
\centering
{\Large Imperial College London\\[0.2cm]
Department Name\\[1.5cm]}

{\huge \textbf{First Line Thesis Title\\Second Line Thesis}\\[1.5cm]}

{\Large \textbf{PhD Thesis}\\[1.0cm]
by\\[0.5cm]
\textbf{Name Surname}\\[1.5cm]


\textbf{Supervisors:}\\
Dr. Name Surname\\
Professor Name Surname\\
Professor Name Surname\\[1.5cm]

{\large Month Day, Year\\[2cm]}

\includegraphics[width=0.25\textwidth]{imperial_logo.png}\\[1cm]
\end{titlepage}
   % Loads your custom title page.
% END SNIPPET: preamble_title

% START SNIPPET: dedication
% -------------------------------
% Dedication Page
% -------------------------------
\thispagestyle{plain}   % Plain page style = page number only, centred at bottom.
\vspace*{5cm}           % Move dedication text downwards. Modify value to adjust placement.

\begin{center}
    \emph{To my family,\\  
    and friends.}
    % Replace with your personal dedication text.
    % Using \emph makes the text italicised.
\end{center}
% END SNIPPET: dedication

% START SNIPPET: Declaration of Originality
% -------------------------------
% Declaration of Originality
% -------------------------------
\clearpage
\thispagestyle{plain}
\vspace*{4cm}

\noindent 
I hereby declare that this thesis and the work presented herein is my own work except where appropriately referenced or acknowledged.
% Mandatory statement confirming originality of the thesis.

\vspace{0.5cm}

\noindent
\textbf{Your Name}   % Replace with your full name.
                      % Good practice: sign physically after printing.

\vspace{5cm}

\noindent
% Required copyright statement (Imperial thesis checklist, Section 6)
The copyright of this thesis rests with the author and is made available under a Creative Commons Attribution-Non Commercial-No Derivatives license…
% END SNIPPET: Declaration of Originality

% START SNIPPET: Abstract
% -------------------------------
% Abstract (MANDATORY)
% -------------------------------
\clearpage
\chapter*{Abstract}
% Imperial requires an abstract of no more than 300 words.
% The abstract provides a concise summary of the thesis.
% END SNIPPET: Abstract

% START SNIPPET: Acknowledgements
% -------------------------------
% Acknowledgements (Optional)
% -------------------------------
\clearpage
\chapter*{Acknowledgements}
% Not mandatory, but standard in most theses.
% Thank supervisors, collaborators, family, funding agencies.
% END SNIPPET: Acknowledgements

% START SNIPPET: Dissemination
% -------------------------------
% Dissemination (Optional)
% -------------------------------
\clearpage
\chapter*{Dissemination}
% Optional section listing publications, posters, conference papers, preprints, etc.
\begin{itemize}
    \item Paper 1: Title, Journal/Conference, Year
    \item Paper 2: Title, Journal/Conference, Year
    \item Poster 1: Title, Conference, Year
\end{itemize}
% END SNIPPET: Dissemination

% START SNIPPET: Nomenclature_Acronyms
% -------------------------------
% Nomenclature (Optional)
% -------------------------------
\clearpage
\printnomenclature
% If the page appears empty: ensure you ran `makeindex` for the nomenclature file.

% -------------------------------
% Acronyms (Optional)
% -------------------------------
\clearpage
\chapter*{Acronyms}
\begin{acronym}[MPC] % Give the longest label here so that the list is nicely aligned
\acro{TLA}{Three Letter Acronym}
\end{acronym}

% Add or modify acronyms in the Acronym.tex file.
% END SNIPPET: Nomenclature_Acronyms

% START SNIPPET: Contents, List of Figures/Tables
% -------------------------------
% Contents, List of Figures/Tables
% -------------------------------
\clearpage
\tableofcontents
% This command tells LaTeX to generate the Table of Contents automatically.
% LaTeX collects all headings (chapters, sections, subsections, etc.) as it compiles and writes them to a helper file with extension .toc. On the next % compilation, it reads that .toc file and uses it to typeset the full table of contents with titles and page numbers.
% If a new section or chapter doesn’t show up in the contents, you usually just need to compile the document at least twice.

\listoffigures
% This command generates a “List of Figures”.
% Every time you use the figure environment together with a \caption{...}, LaTeX records that figure in a helper file with extension .lof. On the next % compilation, it reads that .lof file and prints a list showing each figure number, its caption, and the page number.
% If the list is missing new figures or the page numbers look wrong, compile again so the .lof file is updated and then used.

\listoftables
% This command generates a “List of Tables”.
% Every time you use the table environment with a \caption{...}, LaTeX records that table in a helper file with extension .lot. 
% On the next compilation, it reads that .lot file and prints a list showing each table number, its caption, and the page number.
% As with the other lists, after adding or renumbering tables you usually need at least two compilations for the list to become correct and up to date.

% END SNIPPET: Contents, List of Figures/Tables

% START SNIPPET: CHAPTERS
% -------------------------------
% MAIN CHAPTERS
% -------------------------------
\chapter{Introduction}
\section{Background}
Write the background and motivations here.

\section{Research Questions}
List your research questions.

\subsection{Contributions}
Summarize the main contributions of this thesis.


\chapter{Related Work}
% -------------------------------------------------------------
% Example Tables for the Related Work Chapter
% -------------------------------------------------------------
The Related Work chapter is a crucial component of most theses.  
A practical way to organise your project—and to simplify writing, editing, and version control—is to create a separate \LaTeX{} \texttt{.tex} file for each chapter.  
In this section, we demonstrate how to include and format tables within your thesis, using common academic layouts.
Tables can be a valuable tool in your thesis for a variety of purposes, from presenting results to explaining parameters and organizing complex data.

% -------------------------------------------------------------
% 3×3 Example Table
% -------------------------------------------------------------
As shown in Table~\ref{tab:example_3x3}, the following 3×3 table presents placeholder data for parameters X, Y, and Z.  
This structure is useful for showing compact relationships or comparing small sets of variables.
% START SNIPPET: tables_3x3
\begin{table}[h!]
    \centering
    % The optional caption [short title] defines what appears in the List of Tables.
    \caption[3×3 Example Table]{A 3×3 table template with parameters X, Y, and Z.}
    \label{tab:example_3x3}
    \begin{tabular}{l c r} 
        % l = left aligned, c = centred, r = right aligned
        \toprule
        Parameter & Value & Units \\
        \midrule
        X1 & 10 & m \\
        X2 & 20 & m \\
        X3 & 30 & m \\
        \bottomrule
    \end{tabular}
\end{table}
% END SNIPPET: tables_3x3
% -------------------------------------------------------------
% 5-column Example Table
% -------------------------------------------------------------
Table~\ref{tab:example_5col} shows a 5-column table, suitable for experimental outputs, benchmarking studies, or design specifications.  
You can easily extend the number of columns or replace placeholder values with your own data.
% START SNIPPET: tables_5col
\begin{table}[h!]
    \centering
    \caption[5-column Example Table]{A 5-column table template suitable for experimental results or design specifications with parameters A–E.}
    \label{tab:example_5col}
    \begin{tabular}{l c c c c r}
        % lccccr layout:
        % l = sample name left aligned
        % c = four columns centred
        % r = final column right-aligned (useful for numerical data)
        \toprule
        Sample & A & B & C & D & E \\
        \midrule
        1 & -- & -- & -- & -- & -- \\
        2 & -- & -- & -- & -- & -- \\
        3 & -- & -- & -- & -- & -- \\
        \bottomrule
    \end{tabular}
\end{table}
% END SNIPPET: tables_5col


\chapter{Methods}
% -------------------------------------------------------------
% Example: Writing Mathematical Equations in LaTeX
% -------------------------------------------------------------
In many experimental analyses, it is important to quantify the variability present in the measurements.  
In this section, we demonstrate how to typeset equations in \LaTeX{}, using the sample standard deviation as a simple example.

The standard deviation provides a measure of how spread out the data are from the mean.  
For a sample of $n$ observations, $x_1, x_2, \dots, x_n$, the sample standard deviation $s$ is defined as:
% START SNIPPET: equations_std
\begin{equation}
    s = \sqrt{\frac{1}{n-1} \sum_{i=1}^{n} \left(x_i - \bar{x}\right)^2 }
    \label{eq:std_sample}
\end{equation}

% Explanation:
% - The equation environment automatically numbers the equation.
% - \sum creates the summation symbol.
% - \bar{x} denotes the sample mean.
% - Use \label to refer to the equation later in the text.
% END SNIPPET: equations_std

where $\bar{x}$ is the sample mean, given by:
% START SNIPPET: equations_mean
\begin{equation}
    \bar{x} = \frac{1}{n} \sum_{i=1}^{n} x_i
    \label{eq:mean_sample}
\end{equation}
% END SNIPPET: equations_mean
% Tip:
% If you want the equation unnumbered, use \[
%   ... equation ...
% \] instead of \begin{equation}.

Equation~\ref{eq:std_sample} shows that the standard deviation increases as the data points deviate further from the sample mean.  
This form is commonly used when analysing repeated measurements, sensor data, or experimental uncertainty.



\chapter{Results}
In many experiments, it is useful to present multiple related images together to illustrate comparisons, trends, or different conditions. 
Figure~\ref{fig:four_panels} shows a 2×2 layout with four panels (a–d), which allows the reader to quickly compare results across different experimental settings or data processing steps. 
Each panel can represent a separate condition, time point, or parameter value, while the overall figure provides a cohesive summary.


\begin{figure}[h!]
    \centering
    % Row 1
    \begin{subfigure}[b]{0.4\textwidth}
        \centering
        \includegraphics[width=\textwidth]{phd-thesis/figures/Picture 0.png}
        \caption{Image a}
        \label{fig:sub_a}
    \end{subfigure}%
    \hfill
    \begin{subfigure}[b]{0.4\textwidth}
        \centering
        \includegraphics[width=\textwidth]{phd-thesis/figures/Picture 1.png}
        \caption{Image b}
        \label{fig:sub_b}
    \end{subfigure}
    
    % Row 2
    \begin{subfigure}[b]{0.4\textwidth}
        \centering
        \includegraphics[width=\textwidth]{phd-thesis/figures/Picture 2.png}
        \caption{Image c}
        \label{fig:sub_c}
    \end{subfigure}%
    \hfill
    \begin{subfigure}[b]{0.4\textwidth}
        \centering
        \includegraphics[width=\textwidth]{phd-thesis/figures/Picture 5.png}
        \caption{Image d}
        \label{fig:sub_d}
    \end{subfigure}
    
    % Overall caption
    \caption[Four-panel figure]{Four-panel figure showing different aspects of the experiment. Subfigures a)–d) correspond to different conditions.}
    \label{fig:four_panels}
\end{figure}

When comparing two related results or experimental conditions, it is often convenient to present them side by side in a two-panel figure. 
Figure~\ref{fig:two_panels} shows panels a) and b), which allow the reader to easily observe differences or trends between the two conditions. 
Each panel can represent a distinct variable, time point, or treatment, while the overall figure provides a cohesive comparison.


\begin{figure}[h!]
    \centering
    % Subfigure a
    \begin{subfigure}[b]{0.48\textwidth}
        \centering
        \includegraphics[width=\textwidth]{phd-thesis/figures/Picture 5.png}
        \caption{Description of image a}
        \label{fig:sub_a}
    \end{subfigure}%
    \hfill
    % Subfigure b
    \begin{subfigure}[b]{0.48\textwidth}
        \centering
        \includegraphics[width=\textwidth]{phd-thesis/figures/Picture 2.png}
        \caption{Description of image b}
        \label{fig:sub_b}
    \end{subfigure}
    
    % Overall caption
    \caption[Two-panel figure]{Two-panel figure showing the comparison between condition a) and condition b).}
    \label{fig:two_panels}
\end{figure}


\chapter{Discussion}
\input{chapters/05-discussion.tex}

\chapter{Conclusion}
\input{chapters/06-conclusion.tex}
% END SNIPPET: CHAPTERS

% START SNIPPET: APPENDIX
% -------------------------------
% APPENDIX
% -------------------------------
\appendix
\chapter{Appendix A}
% Add additional appendices as needed: \chapter{Appendix B}, etc.
% END SNIPPET: APPENDIX 

% -------------------------------
% Code Example (minted)
% -------------------------------
% Example of including syntax-highlighted code.
% Make sure shell-escape is enabled in your LaTeX compiler.
% Overleaf → Menu → Settings → Compiler → Enable "Shell escape".
% START SNIPPET: code_pythagoras
\begin{minted}{python}
import math

def pythagoras(a, b):
    return math.sqrt(a**2 + b**2)

side1 = 3
side2 = 4
hypotenuse = pythagoras(side1, side2)
print("Hypotenuse:", hypotenuse)
\end{minted} 
% END SNIPPET: code_pythagoras


% START SNIPPET: Bibliography
% -------------------------------
% Bibliography
% -------------------------------
\backmatter
\printbibliography   % Automatically formats the bibliography using BibLaTeX.
% END SNIPPET: Bibliography

\end{document}
