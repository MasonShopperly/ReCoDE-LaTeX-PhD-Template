\documentclass[11pt, twoside, openany]{book} 
% 'book' provides chapters, front matter, and appendices.
% '11pt' sets the base font size, improves readability for long texts. By modifying 11pt, we can select a new letter size for the full document.
% 'twoside' enables double-sided layout (left/right pages), used for printed theses.
% 'openany' starts chapters on either page (no forced blanks). Use openright for traditional print (chapters on right-hand/odd pages).
% More details about how margins are managed in the document will be explained in the Preamble

% --------------------------------------------------------------------
% PREAMBLE — Global packages and formatting rules
% This section configures fonts, margins, maths, figures, tables,
% code listings, hyperlinks, bibliography formatting, and more.
% It ensures compliance with Imperial College Thesis Requirements
% (Sections 4.1–4.6). Students may add extra packages if needed.
% --------------------------------------------------------------------
% START SNIPPET: preamble
% ------------------------------------------------------
% Encoding and Fonts
% ------------------------------------------------------
\usepackage[utf8]{inputenc}         % Allows UTF-8 characters (e.g., é, ñ, ü).
                                    % Recommended: ensures your thesis handles
                                    % international characters correctly.

\usepackage[T1]{fontenc}            % Improves hyphenation and PDF text extraction.
                                    % Good practice for long academic documents.

\usepackage{lmodern}                % Modern, scalable Latin font family.
                                    % Clean and readable for printed theses.

\usepackage{microtype}              % Improves spacing between letters/words.
                                    % Optional but highly recommended for quality typography.

% ------------------------------------------------------
% Mathematics
% ------------------------------------------------------
\usepackage{amsmath,amssymb}        % Essential math packages for equations, symbols,
                                    % matrices, aligned environments, etc.

% ------------------------------------------------------
% Figures and Captions
% ------------------------------------------------------
\usepackage{graphicx}               % Required to insert images using \includegraphics.
                                    % To scale images: use width=0.5\textwidth, height=..., etc.

\usepackage{caption}                % Controls caption style/spacing.
                                    % To modify caption size: \captionsetup{font=small}

\usepackage{subcaption}             % Allows subfigures (a), (b), etc.
                                    % Useful for 2×2 or side-by-side image layouts.

% ------------------------------------------------------
% Links and PDF Navigation
% ------------------------------------------------------
\usepackage{hyperref}               % Adds clickable links in the PDF.
                                    % Automatically links citations, TOC entries, figures.

\usepackage{bookmark}               % Improves PDF bookmarks stability.
                                    % Best practice for long theses.

% ------------------------------------------------------
% Tables
% ------------------------------------------------------
\usepackage{booktabs}               % High-quality table rules: \toprule, \midrule, \bottomrule.
                                    % For professional-looking tables.

% ------------------------------------------------------
% Acronyms and Nomenclature
% ------------------------------------------------------
\usepackage{acronym}                % For lists of acronyms (optional).

\usepackage{nomencl}                % Creates a nomenclature (list of symbols).
\makenomenclature                   % Generates the nomenclature file.
                                    % Run "makeindex thesis.nlo -s nomencl.ist -o thesis.nls"
                                    % or use Overleaf's built-in tool.

% ------------------------------------------------------
% Page Layout (Margins)
% ------------------------------------------------------
\usepackage{geometry}               
\geometry{
  a4paper,
  inner=3cm, outer=3cm,             % Left/right margins — MUST be symmetrical 
                                    % to satisfy Checklist Section 4.3.
  top=3cm, bottom=3cm
}
                                    % To change margins: modify these values.
                                    % Keep left = right to maintain centred page content.

% ------------------------------------------------------
% Code Listings
% ------------------------------------------------------
\usepackage{minted}                 % For syntax-highlighted code blocks (Python, C++, etc.).
                                    % Usage: \begin{minted}{python} ... \end{minted}
                                    %
                                    % IMPORTANT:
                                    % Overleaf → Menu → Compiler → Enable “Shell escape”.
                                    % Without shell-escape, minted will not compile.

% ------------------------------------------------------
% Bibliography (BibLaTeX + Biber)
% ------------------------------------------------------
\usepackage[
  backend=biber,                    % Biber handles UTF-8, DOIs, URLs, many authors, etc.
  style=authoryear,                 % Produces citations like Smith (2020).
  maxcitenames=2                    % Use “et al.” after 2 authors.
]{biblatex}

\addbibresource{bibliography.bib}   % Main bibliography file.
                                    % Export references from Mendeley, Zotero, etc.
                                    %
                                    % Imperial does NOT mandate a specific citation style.
                                    % Any consistent academic style is acceptable.

% ------------------------------------------------------
% Section Heading Style
% ------------------------------------------------------
\usepackage{sectsty}
\allsectionsfont{\normalfont\scshape}
                                    % Converts section titles to small caps.
                                    % Optional: change to \bfseries or remove styling entirely.

% ------------------------------------------------------
% Custom Commands
% ------------------------------------------------------
\newcommand{\degree}{Ph\ensuremath{.\!}D.\xspace}
                                    % Convenience macro for correctly typesetting "Ph.D."
                                    % You can define similar commands for common terms.
% END SNIPPET: preamble



% Symbols

% -------------------------------
% Page style: centered page numbers at bottom
% -------------------------------
\usepackage{fancyhdr}  
%Loads the fancyhdr package, which gives you full control over page headers and footers.
%Define rich headers/footers with text, page numbers, section/chapter titles, etc.
%Position content using L/C/R (Left, Centre, Right) and O/E (Odd, Even) slots (e.g., LE, RO).
%Add or remove header/footer rules (lines) and control their thickness.
\pagestyle{fancy} %Use the “fancy” style for all subsequent pages.
\fancyhf{} %Clear all header/footer fields.
\fancyfoot[C]{\thepage}
\renewcommand{\headrulewidth}{0pt}
\renewcommand{\footrulewidth}{0pt}

\begin{document}
% -------------------------------
% Title page (page number should appear)
% -------------------------------
\pagenumbering{arabic}
\setcounter{page}{1}
\begin{titlepage}
\centering
{\Large Imperial College London\\[0.2cm]
Department Name\\[1.5cm]}

{\huge \textbf{First Line Thesis Title\\Second Line Thesis}\\[1.5cm]}

{\Large \textbf{PhD Thesis}\\[1.0cm]
by\\[0.5cm]
\textbf{Name Surname}\\[1.5cm]


\textbf{Supervisors:}\\
Dr. Name Surname\\
Professor Name Surname\\
Professor Name Surname\\[1.5cm]

{\large Month Day, Year\\[2cm]}

\includegraphics[width=0.25\textwidth]{imperial_logo.png}\\[1cm]
\end{titlepage}
 %loads the titlepage.tex
% Dedication / Epigraph

\thispagestyle{plain} %Page number centered at the bottom; no header
\vspace*{5cm} 
% Adds vertical space to position the dedication text. Regular \vspace{5cm} might not work at the top of a page, because LaTeX can ignore vertical space there. \vspace*{5cm} forces LaTeX to apply the space even at the top of a page, which is exactly what you want for a dedication page. Adjustable: You can change 5cm to any value (e.g., 3cm, 7cm) depending on how far down you want the dedication text to appear.
\begin{center}
    \emph{To my family,\\  
    and friends.}
    % \emph{...} = emphasized text (typically italics)
    % Replace with your personal dedication text.
\end{center}
% Declaration page
\clearpage
\thispagestyle{plain}
\vspace*{4cm}
\noindent 
I hereby declare that this thesis and the work presented herein is my own work except where appropriately referenced or acknowledged.
% Required by the thesis submission guidelines: include a Statement of Originality at the beginning, confirming the thesis is your own work with all sources properly acknowledged. The text presented here is an example of an acceptable statement.
\vspace{0.5cm}

\noindent
\textbf{Your Name} % It is good practice to include your name in the statement.
\vspace{5cm}

\noindent
% Required by section 6 of the thesis submission checklist: include a copyright statement specifying the Creative Commons license applied to the thesis. The text below serves as an example.
The copyright of this thesis rests with the author and is made available under a Creative Commons Attribution-Non Commercial-No Derivatives license. Researchers are free to copy, distribute or transmit the thesis on the condition that they attribute it, that they do not use it for commercial purposes and that they do not alter, transform or build upon it. For any reuse or distribution, researchers must make clear to others the license terms of this work.

% Abstract
\clearpage
\chapter*{Abstract}
% Required by section 7 of the submission checklist: include an abstract immediately after the title page, limited to a maximum of 300 words. Write your abstract below.

% Acknowledgements
\clearpage
\chapter*{Acknowledgements}  % Creates an unnumbered "Acknowledgements" section.
% Write your acknowledgements here
% Insert personal or professional acknowledgements. Not mandatory but good practice.


% Dissemination page
\clearpage
\chapter*{Dissemination}
% Optional section: A Dissemination page may be added to outline publications, posters, or other outputs generated during the course of the research.
\begin{itemize}
    \item Paper 1: Title, Journal/Conference, Year
    \item Paper 2: Title, Journal/Conference, Year
    \item Poster 1: Title, Conference, Year
\end{itemize}

% Nomenclature
% Nomenclature and Abbreviations
% Nomenclature page
\clearpage
%\chapter*{Nomenclature}  % Manual heading
%\input{Nomenclature.tex} % Loads all \nomenclature definitions
\printnomenclature
% Abbreviations page
\clearpage
\chapter*{Acronyms}
\begin{acronym}[MPC] % Give the longest label here so that the list is nicely aligned
\acro{TLA}{Three Letter Acronym}
\end{acronym}


% Table of Contents and Lists (still in Roman numbering)
\clearpage
\tableofcontents
\listoffigures
\listoftables

% -------------------------------
% Main matter (Arabic numbering)
% -------------------------------


\chapter{Introduction}
\section{Background}
Write the background and motivations here.

\section{Research Questions}
List your research questions.

\subsection{Contributions}
Summarize the main contributions of this thesis.


\chapter{Related Work}
% -------------------------------------------------------------
% Example Tables for the Related Work Chapter
% -------------------------------------------------------------
The Related Work chapter is a crucial component of most theses.  
A practical way to organise your project—and to simplify writing, editing, and version control—is to create a separate \LaTeX{} \texttt{.tex} file for each chapter.  
In this section, we demonstrate how to include and format tables within your thesis, using common academic layouts.
Tables can be a valuable tool in your thesis for a variety of purposes, from presenting results to explaining parameters and organizing complex data.

% -------------------------------------------------------------
% 3×3 Example Table
% -------------------------------------------------------------
As shown in Table~\ref{tab:example_3x3}, the following 3×3 table presents placeholder data for parameters X, Y, and Z.  
This structure is useful for showing compact relationships or comparing small sets of variables.
% START SNIPPET: tables_3x3
\begin{table}[h!]
    \centering
    % The optional caption [short title] defines what appears in the List of Tables.
    \caption[3×3 Example Table]{A 3×3 table template with parameters X, Y, and Z.}
    \label{tab:example_3x3}
    \begin{tabular}{l c r} 
        % l = left aligned, c = centred, r = right aligned
        \toprule
        Parameter & Value & Units \\
        \midrule
        X1 & 10 & m \\
        X2 & 20 & m \\
        X3 & 30 & m \\
        \bottomrule
    \end{tabular}
\end{table}
% END SNIPPET: tables_3x3
% -------------------------------------------------------------
% 5-column Example Table
% -------------------------------------------------------------
Table~\ref{tab:example_5col} shows a 5-column table, suitable for experimental outputs, benchmarking studies, or design specifications.  
You can easily extend the number of columns or replace placeholder values with your own data.
% START SNIPPET: tables_5col
\begin{table}[h!]
    \centering
    \caption[5-column Example Table]{A 5-column table template suitable for experimental results or design specifications with parameters A–E.}
    \label{tab:example_5col}
    \begin{tabular}{l c c c c r}
        % lccccr layout:
        % l = sample name left aligned
        % c = four columns centred
        % r = final column right-aligned (useful for numerical data)
        \toprule
        Sample & A & B & C & D & E \\
        \midrule
        1 & -- & -- & -- & -- & -- \\
        2 & -- & -- & -- & -- & -- \\
        3 & -- & -- & -- & -- & -- \\
        \bottomrule
    \end{tabular}
\end{table}
% END SNIPPET: tables_5col


\chapter{Methods}
% -------------------------------------------------------------
% Example: Writing Mathematical Equations in LaTeX
% -------------------------------------------------------------
In many experimental analyses, it is important to quantify the variability present in the measurements.  
In this section, we demonstrate how to typeset equations in \LaTeX{}, using the sample standard deviation as a simple example.

The standard deviation provides a measure of how spread out the data are from the mean.  
For a sample of $n$ observations, $x_1, x_2, \dots, x_n$, the sample standard deviation $s$ is defined as:
% START SNIPPET: equations_std
\begin{equation}
    s = \sqrt{\frac{1}{n-1} \sum_{i=1}^{n} \left(x_i - \bar{x}\right)^2 }
    \label{eq:std_sample}
\end{equation}

% Explanation:
% - The equation environment automatically numbers the equation.
% - \sum creates the summation symbol.
% - \bar{x} denotes the sample mean.
% - Use \label to refer to the equation later in the text.
% END SNIPPET: equations_std

where $\bar{x}$ is the sample mean, given by:
% START SNIPPET: equations_mean
\begin{equation}
    \bar{x} = \frac{1}{n} \sum_{i=1}^{n} x_i
    \label{eq:mean_sample}
\end{equation}
% END SNIPPET: equations_mean
% Tip:
% If you want the equation unnumbered, use \[
%   ... equation ...
% \] instead of \begin{equation}.

Equation~\ref{eq:std_sample} shows that the standard deviation increases as the data points deviate further from the sample mean.  
This form is commonly used when analysing repeated measurements, sensor data, or experimental uncertainty.



\chapter{Results}
In many experiments, it is useful to present multiple related images together to illustrate comparisons, trends, or different conditions. 
Figure~\ref{fig:four_panels} shows a 2×2 layout with four panels (a–d), which allows the reader to quickly compare results across different experimental settings or data processing steps. 
Each panel can represent a separate condition, time point, or parameter value, while the overall figure provides a cohesive summary.


\begin{figure}[h!]
    \centering
    % Row 1
    \begin{subfigure}[b]{0.4\textwidth}
        \centering
        \includegraphics[width=\textwidth]{phd-thesis/figures/Picture 0.png}
        \caption{Image a}
        \label{fig:sub_a}
    \end{subfigure}%
    \hfill
    \begin{subfigure}[b]{0.4\textwidth}
        \centering
        \includegraphics[width=\textwidth]{phd-thesis/figures/Picture 1.png}
        \caption{Image b}
        \label{fig:sub_b}
    \end{subfigure}
    
    % Row 2
    \begin{subfigure}[b]{0.4\textwidth}
        \centering
        \includegraphics[width=\textwidth]{phd-thesis/figures/Picture 2.png}
        \caption{Image c}
        \label{fig:sub_c}
    \end{subfigure}%
    \hfill
    \begin{subfigure}[b]{0.4\textwidth}
        \centering
        \includegraphics[width=\textwidth]{phd-thesis/figures/Picture 5.png}
        \caption{Image d}
        \label{fig:sub_d}
    \end{subfigure}
    
    % Overall caption
    \caption[Four-panel figure]{Four-panel figure showing different aspects of the experiment. Subfigures a)–d) correspond to different conditions.}
    \label{fig:four_panels}
\end{figure}

When comparing two related results or experimental conditions, it is often convenient to present them side by side in a two-panel figure. 
Figure~\ref{fig:two_panels} shows panels a) and b), which allow the reader to easily observe differences or trends between the two conditions. 
Each panel can represent a distinct variable, time point, or treatment, while the overall figure provides a cohesive comparison.


\begin{figure}[h!]
    \centering
    % Subfigure a
    \begin{subfigure}[b]{0.48\textwidth}
        \centering
        \includegraphics[width=\textwidth]{phd-thesis/figures/Picture 5.png}
        \caption{Description of image a}
        \label{fig:sub_a}
    \end{subfigure}%
    \hfill
    % Subfigure b
    \begin{subfigure}[b]{0.48\textwidth}
        \centering
        \includegraphics[width=\textwidth]{phd-thesis/figures/Picture 2.png}
        \caption{Description of image b}
        \label{fig:sub_b}
    \end{subfigure}
    
    % Overall caption
    \caption[Two-panel figure]{Two-panel figure showing the comparison between condition a) and condition b).}
    \label{fig:two_panels}
\end{figure}


\chapter{Discussion}
\input{chapters/05-discussion.tex}

\chapter{Conclusion}
\input{chapters/06-conclusion.tex}

% -------------------------------
% Appendix
% -------------------------------
\appendix
\chapter{Appendix A}
% Appendix content here

Ensuring the reproducibility of your results is a cornerstone of good scientific practice. There are many ways to present experimental details and computational settings; here, we focus on demonstrating how to include code within your thesis. 

In LaTeX, we can include Python code using the \texttt{minted} package,
which provides nice syntax highlighting. Below is a simple example 
to calculate the hypotenuse of a right triangle using Pythagoras' theorem.

%%%In the following code \begin{minted}   Wraps the  code you want to display. {python} tells minted which language to highlight. Everything inside is shown in the PDF with proper colours and formatting, but it is not executed.
% Code formatting in the PDF Indentation, keywords, comments, and strings are automatically highlighted by minted. This makes the code readable and professional-looking, ideal for tutorials, homework, or papers.

\begin{minted}{python}   

import math

def pythagoras(a, b):
    return math.sqrt(a**2 + b**2)

# Example usage
side1 = 3
side2 = 4
hypotenuse = pythagoras(side1, side2)
print("Hypotenuse:", hypotenuse)
\end{minted}


% -------------------------------
% Back matter
% -------------------------------
\backmatter
\printbibliography

\end{document}
